%===================================================================================
% Chapter: Conclusions
%===================================================================================
\Chapter*{Conclusiones}\label{chapter:conclusions}
%===================================================================================

Esta investigación propone un conjunto de elementos orientados al descubrimiento de conocimiento en textos del lenguaje natural. La propuesta se centra en el idioma español y el dominio de la salud, pero es generalizable en ambos aspectos. Entre las contribuciones fundamentales de esta investigación destacan:

\begin{itemize}
	\item[(1)] La definición de un modelo de anotación de propósito general que logra capturar los rasgos semánticos más relevantes contenidos en documentos de texto plano. El mismo es usado como base en la construcción de la ontología propuesta.
	\item[(2)] La definición de un formato de anotación de archivos para el esquema conceptual previamente definido.
	\item[(3)] Se diseñó una propuesta de ontología donde se puede representar un corpus de documentos escritos en lenguaje natural.
	\item[(4)] Se implementó un algoritmo computacional para representar un corpus anotado como grafo de conocimiento a través de dicha ontología.
\end{itemize}

A menudo, una ontología de un dominio no es un objetivo en sí misma. Desarrollarla es similar a definir un conjunto de datos y su estructura para que los utilicen otros programas. Los métodos de resolución de problemas, las aplicaciones independientes del dominio y los usuarios, a menudo las utilizan como datos, en conjunto con bases de conocimiento creadas a partir de las mismas. Por ejemplo, en esta investigación se desarrolla una ontología de dominio médico, la cual se puede utilizar como base para algunas aplicaciones que ofrezcan un conjunto de herramientas de gestión de la salud.

En los resultados se demostró el descubrimiento de conocimiento implícito en el corpus. Esto trae consigo disímiles ventajas, desde el propio descubrimiento de este conocimiento, hasta la interpretación y aprendizaje de un gran número de textos escritos en lenguaje natural, en apenas segundos, mediante el uso de un equipo de cómputo y las propuestas ofrecidas en esta investigación. El conocimiento extraído podría ser usado posteriormente por especialistas en el tema o usuarios, y de esta manera ahorrar el tiempo que tomaría la lectura e interpretación del propio corpus usado para esta tarea.

Teniendo en cuenta que un humano debe tener una gran capacidad de memorización para poder recordar todo lo aprendido en un corpus de documentos, la traba que puede ocasionar tenerlo escrito en un idioma que no se domine, y el factor de no olvidar lo aprendido de él al pasar el tiempo, es clave la utilización de un equipo de cómputo para la creación de la base de conocimiento respectiva al corpus. La misma puede ser fácilmente guardada, leída y usada en el propio sistema o incluso en otros, aportando gran versatilidad al uso de las técnicas mostradas en este estudio.

El descubrimiento automático de conocimiento en el dominio médico tiene especial importancia, pues permitiría identificar interacciones ocultas en la literatura. Además, a pesar de que los recursos médicos disponibles en idioma español son abundantes, los recursos necesarios para la creación de sistemas de extracción automática son más escasos que en otros idiomas, por lo cual la construcción de una ontología y un grafo de conocimiento basado en un corpus del propio dominio, constituyen un hecho relevante para el desarrollo de nuevos sistemas y la continuación de esta investigación en un futuro.