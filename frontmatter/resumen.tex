%===================================================================================
% Chapter: Abstract (spanish)
%===================================================================================
\ChapterOutOfToc{Resumen}\label{chapter:abstract_spanish}
%===================================================================================

En los últimos años se ha evidenciado un aumento en el desarrollo de técnicas para descubrir conocimiento de forma automática en documentos escritos en lenguaje natural. El procesamiento automático va aparejado a la posibilidad de analizar colecciones de información con disímiles textos. En el área de la medicina, el auge de estas tecnologías es especialmente significativo, pues permite favorecerse de la gran cantidad de información disponible para el avance de este campo, que posee gran importancia para la sociedad. Por otra parte, estas técnicas suelen apoyarse en corpus anotados, los cuales son recursos escasos. Esto se vuelve crítico en el idioma español, donde la cantidad existente es ínfima y de dominio menos generalizado.

En este estudio se define un modelo de anotación de propósito general con el objetivo de capturar los rasgos semánticos más relevantes en los documentos de texto. Además, se presenta un esquema de ontología que se usará para la extracción de conocimiento de forma automática. También se ofrecen los pasos a seguir para la implementación de un algoritmo computacional que busca representar un corpus anotado como un grafo de conocimiento, siguiendo las reglas definidas por la propia ontología. Por último, se muestran las tareas realizadas para la validación de las propuestas dadas, así como resultados en términos matemáticos.

Los resultados alcanzados demuestran que el descubrimiento de conocimiento constituye un campo de investigación activo, donde pueden aplicarse técnicas de aprendizaje automático logrando resultados positivos. Se propone la verificación y comparación de un grafo de conocimiento específico creado a partir de las propuestas brindadas en este estudio respecto a la capacidad de aprendizaje e interpretación de un grupo de expertos en el mismo tema. Además se ofrece la continuación de esta línea de investigación con el objetivo de mejorar la efectividad de las propuestas dadas y su aplicación en otros dominios.