%===================================================================================
% Chapter: Ontology Learning
%===================================================================================
\chapter{Aprendizaje de Ontologías}\label{chapter:ontology-learning}
%===================================================================================
El presunto desarrollo tecnológico se ha exacerbado con el advenimiento cada vez mayor del uso del Internet y otros medios avanzados efectivos que garantizan un mejor futuro para cuestiones importantes de la vida. Debido al continuo aumento del flujo informativo se hace cada vez más necesario el uso de herramientas que permitan identificar, capturar y representar el conocimiento del dominio dentro de los sistemas de información.

Para ello, ciencias avanzadas como la Ciencia de la Computación y de la Comunicación comprenden la ontología como esa vía de formal de tipos, propiedades, y relaciones entre entidades que realmente o fundamentalmente existen para un dominio de discurso en particular. Estas son creadas para limitar la complejidad de cualquier tema y para organizar la información, es una medida eficaz en la soluciones de problemas comunes en la vida diaria, que debido a la sobreinformación no se podrían llevar a cabo de forma manual.

Por su parte, permiten crear entendimiento compartido al unificar los diferentes puntos de vista que sirven para facilitar la comunicación entre los actores implicados en la construcción de sistemas de informaciones referidos al tema en cuestión, permitir el reuso del conocimiento del dominio, pues sirve de base ya creada para posteriores investigaciones, facilitar la recuperación, integración e interoperatividad entre fuentes de conocimiento heterogéneas. Se provee una base para la representación del conocimiento del dominio y ayudar a identificar las categorías semánticas del mismo. (REFERENCIAS) D.Pisanelli; A. GANGEMI; G:Steve Ontologies and Information Systems: The Marriage of the Century https://www.loa-cnr.it/Papers/lyee.pdf

Surge entonces el creciente interés de estudiar técnicas para el descubrimiento automático de conocimiento. El procesamiento automático trae consigo la posibilidad del analizar colecciones masivas de información. Sin embargo, la mayor parte de estas colecciones almacenan la información disponible en documentos textuales escritos en lenguaje natural. La naturaleza en que se expresa la información y su estructura semántica poco unificada, se vuelven la principal fuente de retos para el procesamiento automático. Entre los retos destacan la ambigüedad del lenguaje natural y el número de idiomas en que puede estar escrito. (Juan Pablo Consuegra) Tesis de Licenciatura

Falta la parte más del tema específico

Problemática
Objetivos
General
El objetivo principal propuesto para cumplir en la investigación es crear una ontología para representar el conocimiento descrito en un corpus anotado.  
Específicos
- Estudiar los esquemas de anotación y corpus usados en diversas tareas de extracción del conocimiento.
- Definir un esquema conceptual generalizable.
- Diseñar propuesta de ontología donde se pueda representar el corpus.
-Implementar un marco experimental para la evaluación de la propuesta de solución.
-Comparar la propuesta con las anteriormente brindadas y ofrecer resultados.  